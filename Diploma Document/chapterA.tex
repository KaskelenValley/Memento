\chapter{Literature Review}\label{ch:A}
Nowadays emotional wellbeing and mental health are underestimated but have a crucial meaning. Recent studies showed that Kazakhstan has one of the highest suicide rates \cite{WOS:000670909100001}. But such a high rate is caused not only due to labor, financial and economic factors but also because people don not treat their mental health as equally as physical. Also, as it will be shown down below mental health at a bad state might be one of the reasons of MCI (Mild Cognitive Impairments) and later dementia. This chapter concentrates on the research studies that relate to the mental health, mild cognitive impairments, mood disorders and the use of technologies in tracking those factors. Later on taking on an account the results of these studies we made a decision to form our solution in the way that directed our developing app in the course of mood tracking by the use of audio diaries.

\section{Mood and mild cognitive impairments}
Mood is a central aspect of mental health. Mood defines not only the current state of emotional being but also might cause serious issues. The systematic review with meta-analysis conducted that symptoms of depression and anxiety were more prevalent in people with MCI than in people with normal cognitive function, and increased the risk of progression from no cognitive impairment to MCI \cite{yates_clare_woods_2013}. The study showed results regarding the effect of such symptoms on progression from MCI to dementia. Other study explores the relationship between behavior and daily mood of older adults with different levels of cognitive impairment across four weeks. The sample included persons with early stage of Alzheimer's disease (AD), persons with MCI and cognitively healthy persons (CH) \cite{WOS:000346995600002}. AD and MCI adults showed lower mood than the CH group which leads to the conclusion that that there is a correlation between the mood and cognitive status.

As previous study showed mood might relate to the cognitive status. Sleep quality also relates to emotional and cognitive health. Sensor-based sleep and mood monitoring systems promise to prolong independent living of elderly people with declining physical and cognitive functions \cite{WOS:000591565600071}. 

That is why it is important to track the mood. There is a study where 22 participants who had used mood-tracking apps were interviewed using a semistructured interview and card sorting task \cite{schueller2021understanding}. The study showed that users of mood-tracking apps were primarily motivated by negative life events or shifts in their own mental health that prompted them to engage in tracking and improve their situation. As a result using a mood-tracking app facilitated self-awareness and helped them to look back on a previous emotion or mood experience to understand what was happening. However, some users reported less inclination to document their negative mood states and preferred to document their positive moods. To avoid that we offer a solution with automatic mood definition.

But tracking the mood can be one of the challenges due to its subjective sources. Research on emotion is full of methodological limitations, as feelings can have non-discrete, ephemeral, and ineffable qualities. Mobile technology makes it possible to extend mood self-assessment from lab to real life rather, collecting mood data frequently, over long time, in variety of life situations \cite{khue2015mood}. There is a study focused on designing an app as a self-healing tool aiming to engage people in a series of music collecting activities plus journaling via a tangible recording audio book. These activities are informed by existing psychological evidences on music therapy \cite{WOS:000723951900034}. While the authors focused not on mood tracking but rather healing through the music the other ones made a research on helping pregnant women with depressive symptomatology with the MTA app. The MTA app monitored activity, assessed mood and alerted obstetric providers of signs of worsening mood. Women who received telephone contact from a provider triggered by an MTA app alert were significantly more likely to receive a mental health specialist referral  \cite{hantsoo2018mobile}.

Audio diaries provide a method for capturing the sequential and varied experience of emotions as they emerge from everyday life \cite{cottingham2020capturing}. The study shows how audio diaries might be used to capture sudden emotions that appears spontaneously and may reflect infamous or negative social views and experiences and processes of emotional reflexivity of everyday life.

\section{Audio diary}
Audio diaries offer a highly convenient means to capture real-time experiences and provide a rich record of conversational narratives \cite{sawhney2018audio}. 
Audio diaries makes it possible to provide a method where young people express detailed reflections on their day-to-day encounters as well as ordinary silenced topics, including hidden emotions \cite{mupambireyi2019reflections}.  In the another research young people recorded their experiences and reflections about growing up on microcassette recorders \cite{worth2009making}. The explored studies have a common conclusion: audio diaries tend to reveal people's hidden feelings and make them openly talk about their deepest secrets which might release the psychological tension. Also, self-reflection helps to revise the feeling and thus makes it possible to more easily experience negative emotions.