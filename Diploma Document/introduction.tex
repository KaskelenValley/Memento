\chapter{Introduction}\label{ch:intro}
%these sections are optional, up-to the author
\section{Motivation}
Nowadays, people are prone to depression due to many factors in this rapidly changing world. Social media, bad news, and constant information noise can cause at least apathy and frustration. Furthermore, the bulk of human beings do not self-reflect and are not aware of their state of mental health. However, in our country, not enough attention is paid to identifying and solving problems associated with mental illness, such as Alzheimer's disease or even a high suicide rate.

In Kazakhstan, people suffer from dementia ten times more often than official statistics admit. While the government ignores the growth of citizens with cognitive impairments, the disease becomes a serious social and ethical issue \cite{kz-dementia}. Results from the national survey of root suicide causes showed that teen suicide is very strongly associated with the poor mental health of children and adolescents. Health disorders such as depression, anxiety, and fear are frequently detected when examining the psychological state of minors \cite{postcovid-suicide}.

These data suggest that the state of mental health of a person is practically not taken into account, although the problem under consideration is the scourge of our time.

According to a study in 2018 \cite{gbd-results}, testees who executed emotional-focused journaling were inclined to show a low level of mental distress and increased well-being. The study likewise showed a link between journaling and improved immune systems in patients whose lymphocyte counts increased after writing their thoughts.

Overall, we can conclude that even minor daily habits, such as keeping track of emotions in writing, might significantly prevent people from mental diseases.

\section{Aims and Objectives}
Based on the conclusions made in the previous paragraphs, we have set the following aims, which reflect what we hope to achieve, and the objective to achieve these aims.

    \textbf{Aim}:
To develop the audio journaling application Memento for people with cognitive impairments.

\textbf{Objectives}:
\begin{itemize}
\item To extract the data from the prominent scientific records of the display of cognitive impairments and their consequences;
\item To conduct a customer development interview and develop a respective user-friendly interface for the application;
\item To choose an actual methodology and stack of the technology for further app development;
\item To synthesize the collected data and create the Memento application;
\item To conduct comprehensive testing of the system on the target audience.
\item To identify new features for further improvement of the application.
\end{itemize}

\section{Thesis Outline}
The first chapter is the \nameref{ch:intro} chapter. It provides insights into the motivation of the thesis, the aims and objectives set, and a brief overview of subsequent chapters.

Chapter 2 is for \nameref{ch:A} and formulating the problem to solve.

The third chapter describes the solution to the problem and its methodology. It explains what methods we intend to use in researching and developing the application.

Chapter 4 provides insight into the implementation of the project and the contributions of each team member through a detailed specification of the application's structure.

The conclusion chapter emphasizes the importance of the developed system and further intentions to promote the Memento application.
